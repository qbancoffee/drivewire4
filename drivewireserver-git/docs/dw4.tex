%
%
%
% dw4.tex
% J. Bilmes



\documentclass{article}
\usepackage{times,bgp,listings}


\title{DriveWire 4 Specification}
\author{Boisy G. Pitre
\thanks{The author acknowledges the support of Howard Chizeck in
making this project a reality},
Aaron Wolfe\\
{boisy@boisypitre.com, aawolfe@gmail.com}\\
\\
% Place other locations here if you need them by uncommenting
% the following lines.
%\\
%Atomic Computer \& Engineering Facility \\
%Las Vegas NV, 034343\\
}



% Change to the current month of the series
\reportmonth{May}
% Change to the current year of the series
\reportyear{2010} 
% Change to the TR number that you obtained from the
% UWEETR web pages when you initially created a new
% TR number. Only provide the last 4 digits here, the year
% goes in the \reportyear{} field above.
\reportnumber{0000}

\begin{document}

% This first line makes the cover page, which prints the TR number.
\makecover
% This second line makes the title portion of the first page.
\maketitle

\begin{abstract}

DriveWire has evolved as a first-class environment for connecting the TRS-80/Tandy Color Computer to a modern computing system for the purposes of sharing information and data. DriveWire 4 is the fourth generation specification, adding networking and printing capabilities.

\end{abstract}


\section{Introduction} 

The DriveWire Protocol Specification defines a set of communication messages that allow the Tandy/Radio Shack Color Computer (CoCo) to utilize the storage capacity of a modern personal computer (Server). This storage model gives the CoCo the appearance that it is directly connected to a large storage device, when in fact, connectivity is achieved using a serial cable connected from the Server to the CoCo�s built-in �Serial I/O� port.

There are distinct advantages in utilizing DriveWire as a storage medium:

\begin{enumerate}
\item Cost and space savings: no additional hardware is needed for a CoCo to use mass storage; no floppy controller, hard drive controller or hard drives are required.

\item Remote operation: the serial cable that tethers the CoCo to the Server can extend for some length, allowing the CoCo to be positioned a considerable distance from the Server.

\item Easy data management: virtual disks that reside on the Server can be easily copied, emailed or archived for backup purposes.

\end{enumerate}

The essence of communication between the CoCo and Server is a documented set of uni- and bi-directional messages called transactions.	Each transaction is composed of one or more packets, which are passed between CoCo and Server through a serial line connection.	Using clever timing techniques on the CoCo, baud rates of up to 115,200 are reliably achievable.

\section{System Requirements}

The Server shall be defined as a modern personal computer or hand-held device with sufficient mass storage and an RS-232 serial port that can be reliably driven at rates of up to 115,200 bits per second.

The CoCo shall be defined as any model of Color Computer 2 or greater, produced by Tandy/Radio Shack with at least 16K of RAM, as well as an appropriate operating environment that provides mass storage services (such as Disk BASIC or OS-9).

\section{Physical Interface Requirements}

The physical cable connecting the CoCo to the Server is recommended to be no longer than 10 feet in length. At a minimum, it shall carry the following four wires:
\begin{enumerate}
\item Ground
\item Vcc
\item RD (Read Data)
\item WD (Write Data)
\end{enumerate}
A 4 pin DIN connector shall be on one end of the cable which will mate to the jack on the back of the CoCo marked �Serial I/O�.	The other end of the cable shall be either a DB-9 or DB-25 female connector which will mate to an appropriate serial port on the Server.


The following table shows the connections between the CoCo 4 pin DIN connector to either a 9-pin DB-9 serial connector or a 25-pin DB-25 serial connector.

\begin{table}[ht]
\caption{Pin-out}
\begin{center}
\begin{tabular}{|lll|}
\hline
CoCo 4-pin DIN & DB-9 & DB-25 \\ \hline
Pin 2 (RD) & Pin 3 (TD) & Pin 2 (TD) \\
Pin 3 (GND) & Pin 5 (GND) & Pin 7 (GND) \\
Pin 4 (TD) & Pin 2 (RD) & Pin 3 (RD) \\
\hline
\end{tabular}
\end{center}
\end{table}

\section{Communication Protocol}

The communication protocol shall follow the RS-232 standard of 8-N-1 (1 start bit, 8 data bits, and 1 stop bit). For the Color Computer 2, the supported clock speed shall be .89MHz and the bit rate shall be 57,600 bits per second. For the Tandy Color Computer 3, the supported clock speed shall be 1.78MHz and the bit rate shall be 115,200 bits per second.

\section{Transactions}

The transaction is the basis for exchanging information between CoCo and Server.	 All transactions consist of one or more packets of information, and are generally initiated from the CoCo.
Some transactions are uni-directional: from the CoCo to the Server.	Other transactions are bi-directional. In a bi-directional transaction, both the CoCo and the Server shall send their respective responses within 250 milliseconds (1/4 of a second) of receiving the last packet from its peer, or else each shall presume a timeout condition and abort the transaction.  Drivewire 4 introduces a variable rate CoCo initiated poll operation when used with NitrOS-9.   This poll is primarily used to retrieve incoming data for the virtual serial ports.  However, the server response to the poll request can effectively be used as a mechanism for limited server initiated communications.


With few exceptions, all CoCo initiated packets have the following common header:

\begin{table}[ht]
\caption{CoCo Initiated Packet Header}
\begin{center}
\begin{tabular}{|ll|}
\hline
Byte & Value \\ \hline
0 & Command Code \\
\hline
\end{tabular}
\end{center}
\end{table}

The command code is a byte value that communicates to the Server the desired command to be executed.

\subsection{The Initialization Transaction}

This is a uni-directional transaction, and is performed by the CoCo as a notification to the Server when its driver software is initialized. The following Initialization packet is sent to the Server:

\begin{table}[ht]
\caption{Initialization Packet}
\begin{center}
\begin{tabular}{|ll|}
\hline
Byte & Value \\ \hline
0 & OP\_INIT (\$49) \\
\hline
\end{tabular}
\end{center}
\end{table}

Upon receipt of the packet, the Server shall perform the following steps:

\begin{enumerate}
\item If not already done, open the disk images for all supported drives.
\end{enumerate}

\subsection{The Read Extended Transaction}

This bi-directional transaction is initiated by the CoCo to request one 256 byte sector of data from the Server. When the CoCo desires to request a sector, it will send the following Read Extended Request packet:

\begin{table}[ht]
\caption{Read Extended Request Packet}
\begin{center}
\begin{tabular}{|ll|}
\hline
Byte & Value \\ \hline
0 & OP\_READEX (\$D2) \\
1 & Drive number \\
2 & Bits 23-16 of LSN \\
3 & Bits 15-8 of LSN \\
4 & Bits 7-0 of LSN \\
\hline
\end{tabular}
\end{center}
\end{table}

The drive number is a value between 0-255, and represents the disk image where the desired sector is located. The LSN is the Logical Sector Number, and is a 24 bit value.

\subsubsection{Read Extended Packet}

Upon receipt of the packet, the Server shall consult the appropriate disk image for the provided LSN and read that 256 byte sector. If an error occurs and the Server cannot read the specified sector, it shall return the Read Failure packet:

\begin{table}[ht]
\caption{Read Failure Packet}
\begin{center}
\begin{tabular}{|ll|}
\hline
Byte & Value \\ \hline
0-255 & 256 bytes of zeroed sector data \\
\hline
\end{tabular}
\end{center}
\end{table}

If no error occurred on reading the sector from the Server file system, the Server shall return the following packet:

\begin{table}[ht]
\caption{Read Success Packet}
\begin{center}
\begin{tabular}{|ll|}
\hline
Byte & Value \\ \hline
0-255 & 256 bytes of valid sector data \\
\hline
\end{tabular}
\end{center}
\end{table}

Upon receipt of either the Read Success or the Read Failure packet, the CoCo shall:

\begin{enumerate}
\item Compute the 16 bit checksum on the 256 bytes of sector data received from the Server.
\item Send the computed checksum to the Server as follows:

\begin{table}[ht]
\caption{Computed Checksum Packet}
\begin{center}
\begin{tabular}{|ll|}
\hline
Byte & Value \\ \hline
0 & Bits 15-8 of the checksum (computed by CoCo) \\
1 & Bits 7-0 of the checksum (computed by CoCo) \\
\hline
\end{tabular}
\end{center}
\end{table}

\end{enumerate}

Upon receipt of the checksum packet, the Server shall take one of two actions:

\subsubsection{Action 1 - No Error Condition}

If the server acquired the requested sector without error, then it shall compute the 16 bit checksum on the 256 bytes of sector data sent to the CoCo and compare its calculated checksum against that of the CoCo.

If the compare yields a match occurs, then Server shall send the following packet to the CoCo:

\begin{table}[ht]
\caption{No Error Packet}
\begin{center}
\begin{tabular}{|ll|}
\hline
Byte & Value \\ \hline
0 & No error code (0) \\
\hline
\end{tabular}
\end{center}
\end{table}

If the compare does not yield a match, then Server shall send the following packet to the CoCo:

\begin{table}[ht]
\caption{Checksum Mismatch Packet}
\begin{center}
\begin{tabular}{|ll|}
\hline
Byte & Value \\ \hline
0 & Checksum mismatch (243) \\
\hline
\end{tabular}
\end{center}
\end{table}

Once either of these packets is received by the CoCo, the Read transaction is considered terminated.

\subsubsection{Action 2 - Error Condition}

If the Server was unable to acquire the requested sector due to an error, then the Server shall return an appropriate read error code:

\begin{table}[ht]
\caption{Read Error Packet}
\begin{center}
\begin{tabular}{|ll|}
\hline
Byte & Value \\ \hline
0 & Read error code (1-242 or 244-255) \\
\hline
\end{tabular}
\end{center}
\end{table}

Once the Read Error packet is received by the CoCo, the Read transaction is considered terminated.

See Appendix A for a list of supported error codes, and Appendix B for the checksum algorithm used by the CoCo to generate the checksums.

\subsection{ReRead Extended Packet}
In the event that the CoCo receives a Checksum Mismatch packet during the Read Transaction, the CoCo MAY send the following packet to the Server:

\begin{table}[ht]
\caption{ReRead Packet}
\begin{center}
\begin{tabular}{|ll|}
\hline
Byte & Value \\ \hline
0 & OP\_REREADEX (\$F2) \\
1 & Drive number \\
2 & Bits 23-16 of LSN \\
3 & Bits 15-8 of LSN \\
4 & Bits 7-0 of LSN \\
\hline
\end{tabular}
\end{center}
\end{table}

Upon receiving this packet, the Server will attempt a retransmission of the requested sector.

At any time, the CoCo may suspend the re-reading of a sector if errors continue to accumulate, and the server will return to monitoring for additional commands.

\section{The Write Transaction}
This bi-directional transaction allows the CoCo to send one 256 byte sector of data to the Server in order for it to be written to a specific virtual drive.

\begin{table}[ht]
\caption{Write Packet}
\begin{center}
\begin{tabular}{|ll|}
\hline
Byte & Value \\ \hline
0 & OP\_WRITE (\$57) \\
1 & Drive number \\
2 & Bits 23-16 of LSN \\
3 & Bits 15-8 of LSN \\
4 & Bits 7-0 of LSN \\
5-260 & 256 bytes of sector data to write \\
261 & Bits 15-8 of checksum (computed by CoCo) \\
262 & Bits 7-0 of checksum (computed by CoCo) \\
\hline
\end{tabular}
\end{center}
\end{table}

The Server shall implement a timeout of 250 milliseconds (1/4 second), in the event that the CoCo stops sending data. This allows the Server to recover from the Write Transaction state and return to processing other packets.

Once the Server receives this packet, it shall:
\begin{enumerate}
\item Compute the 16 bit checksum over the 256 byte sector it received
\item Compare the checksum that it computed against the checksum provided by the CoCo
\end{enumerate}

Once the checksums have been compared, the Server shall acknowledge the transmission as follows:

\begin{table}[ht]
\caption{Checksum OK Packet}
\begin{center}
\begin{tabular}{|ll|}
\hline
Byte & Value \\ \hline
0 & 0 (checksum OK) \\
\hline
\end{tabular}
\end{center}
\end{table}

The CoCo, upon receipt of the packet, will do one the following:

\begin{enumerate}
\item If the error code received is 0 (\$00), the CoCo will assume the transfer of the sector was a success and terminate the transmission.
\item If the error code received is 243 (E\$CRC), the CoCo MAY send the following packet:

\begin{table}[ht]
\caption{ReWrite Packet}
\begin{center}
\begin{tabular}{|ll|}
\hline
Byte & Value \\ \hline
0 & OP\_REWRITE (\$77) \\
1 & Drive number \\
2 & Bits 23-16 of LSN \\
3 & Bits 15-8 of LSN \\
4 & Bits 7-0 of LSN \\
5-260 & 256 bytes of sector data to write \\
261 & Bits 15-8 of checksum (computed by CoCo) \\
262 & Bits 7-0 of checksum (computed by CoCo) \\
\hline
\end{tabular}
\end{center}
\end{table}

\end{enumerate}

And the Write Transaction will commence again.

\subsection{GetStat/SetStat Transaction}
This uni-directional transaction is for informational purposes only and may not be supported in all client environments. Its inclusion into the protocol specification is mainly for the benefit of DriveWire-enabled OS-9 drivers that execute on the CoCo.

In such a driver, all GetStats/SetStats are passed to the Server via the OP\_GETSTAT or OP\_SETSTAT code, followed by a byte representing the GetStat/ SetStat code. The Server may wish to log this information, but does not act upon the packet itself.

\begin{table}[ht]
\caption{GetStat/SetStat Packet}
\begin{center}
\begin{tabular}{|ll|}
\hline
Byte & Value \\ \hline
0 & OP\_GETSTAT (\$47) or OP\_SETSTAT(\$53) \\
1 & Drive number \\
2 & GetStat or SetStat code \\
\hline
\end{tabular}
\end{center}
\end{table}

\subsection{Termination Transaction}
Upon termination of the DriveWire client on the CoCo, this uni-directional transaction shall be sent to the Server:

\begin{table}[ht]
\caption{Terminate Packet}
\begin{center}
\begin{tabular}{|ll|}
\hline
Byte & Value \\ \hline
0 & OP\_TERM (\$54) \\
\hline
\end{tabular}
\end{center}
\end{table}

Upon receipt of the packet, the Server should:

\begin{enumerate}
\item Close the disk file associated for all supported drives.
\end{enumerate}

\subsection{The Reset Transaction}

When the CoCo�s reset button is pressed, the state of the serial port changes such that byte \$FE (OP\_RESET1) or \$FF (OP\_RESET2) will be present on the serial line. This condition is therefore considered a uni-directional transaction to the Server to indicate a global reset:

\begin{table}[ht]
\caption{Reset Packet}
\begin{center}
\begin{tabular}{|ll|}
\hline
Byte & Value \\ \hline
0 & OP\_RESET1 (\$FE) or OP\_RESET2 (\$FF) \\
\hline
\end{tabular}
\end{center}
\end{table}

Upon receipt of the packet, the Server shall:

\begin{enumerate}
\item Reset any statistical values related to read/write and sector transfer throughput for ALL drives.
\item Flush all caches connected to the virtual drives.
\end{enumerate}

\subsection{No-Op Transaction}

This is a uni-directional transaction that can be ignored by the Server.

\begin{table}[ht]
\caption{No-Op Packet}
\begin{center}
\begin{tabular}{|ll|}
\hline
Byte & Value \\ \hline
0 & OP\_NOP (\$00) \\
\hline
\end{tabular}
\end{center}
\end{table}

\subsection{Time Transaction}

This bi-directional transaction allows the CoCo to request time and date information from the Server:

\begin{table}[ht]
\caption{Time Packet}
\begin{center}
\begin{tabular}{|ll|}
\hline
Byte & Value \\ \hline
0 & OP\_TIME (\$23) \\
\hline
\end{tabular}
\end{center}
\end{table}

Upon receipt of the packet, the Server shall return the following packet to the CoCo:

\begin{table}[ht]
\caption{Time Response Packet}
\begin{center}
\begin{tabular}{|ll|}
\hline
Byte & Value \\ \hline
0 & Current year minus 1900 (0-255) \\
0 & Month (1-12) \\
0 & Day (1-31) \\
0 & Hour (0-23) \\
0 & Minute (0-59) \\
0 & Second (0-59) \\
0 & ??? Day of week (0-6; 0 = Sunday) \\
\hline
\end{tabular}
\end{center}
\end{table}

\section{Printing}

DriveWire 3 introduces a special set of transactions to allow printing to the Server that it is connected to. The Server shall maintain a printer buffer where data bytes are added until a flush command is sent.	Upon receiving the flush command, the Server shall prompt the user for a printer to send the buffered data to.

\subsection{Print Transaction}
Using this uni-directional transaction, the CoCo notifies the Server that it has a single byte of data to add to the print queue.

\begin{table}[ht]
\caption{Print Packet}
\begin{center}
\begin{tabular}{|ll|}
\hline
Byte & Value \\ \hline
0 & OP\_PRINT (\$50) \\
1 & Byte of data to add to the queue \\
\hline
\end{tabular}
\end{center}
\end{table}

Upon receiving this packet, the Server shall add the passed byte to its internal print buffer.

\subsection{Print Flush Transaction}

Using this uni-directional transaction, the CoCo notifies the Server that it has completed sending printer data and that the Server should send its print buffer to the printer.

\begin{table}[ht]
\caption{Print Flush Packet}
\begin{center}
\begin{tabular}{|ll|}
\hline
Byte & Value \\ \hline
0 & OP\_PRINTFLUSH (\$46) \\
\hline
\end{tabular}
\end{center}
\end{table}

Upon receiving the flush command, the Server shall prompt the user for a printer to print the buffer to.


\section{Virtual Serial}

DriveWire 4 introduces a special set of transactions to support up to 15 virtual serial devices.  These devices are presented by the NitrOS-9 drivers as more or less standard serial ports which can be used by standard terminal software or BBS systems.  The DriveWire server software can emulate a standard Hayes modem that accepts Internet hostnames or IP addresses in place of phone numbers in the ATD command.  The server also provides an API that can be used to establish incoming or outgoing TCP connections with a variety of server side processing options.  Additional API commands can be used to query server status information or modify the server's operating parameters.

Because the CoCo does not listen for server initiated communications while in standard operating mode, the CoCo must poll the server for incoming serial data.  In order to provide acceptable latency in interactive applications without burdening the system, these polls are designed to use as few bytes as possible.  The rate at which these polls are performed is dynamically adjusted to meet current traffic requirements with the slowest possible polling frequency.

\subsection{SerPoll transaction}
Using this bi-directional transaction, the CoCo queries the Server for queued incoming serial data.  The server responds with 2 bytes which can indicate a range of conditions and may contain one byte of data.

\subsection{SerReadM transaction}
Using this bi-directional transaction, the CoCo requests a fixed number of bytes from the input queue of a specific port.

\subsection{SerWrite transaction}
Using this uni-directional transaction, the CoCo sends one byte to the output queue of a specific port.

\subsection{SerInit transaction}
Using this uni-directional transaction, the CoCo notifies the server that a specific port has been initialized by NitrOS-9.

\subsection{SerTerm transaction}
Using this uni-directional transaction, the CoCo notifies the server that a specific port has been terminated by NitrOS-9.

\subsection{SerSetStat transaction}
Using this uni-directional transaction, the CoCo sends various status information about a specific port to the Server.

\subsection{SerGetStat transaction}
Using this uni-directional transaction, the CoCo notifies the server that various GetStat calls have been issued to a specific port's driver.


\section{Wirebug}
DriveWire 3 introduces a special set of transactions to facilitate remote debugging between a CoCo and the Server that it is connected to. This feature, known as WireBug, allows the transfer of register information and memory from the CoCo to the Server.	Once the CoCo has entered WireBug mode, it can receive messages initiated by the server.

Because of the nature of the protocol and to increase accuracy and efficiency, WireBug uses a fixed packet size of 24 bytes and reserves the last byte as an 8- bit checksum byte.

\subsection{Enter WireBug Mode Transaction}
Using this uni-directional transaction, the CoCo notifies the Server that it has entered WireBug remote debugging mode.

\begin{table}[ht]
\caption{WireBug Mode Packet}
\begin{center}
\begin{tabular}{|ll|}
\hline
Byte & Value \\ \hline
0 & OP\_WIREBUG\_MODE (\$42) \\
1 & CoCo Type (\$02 = CoCo 2, \$03 = CoCo 3) \\
2 & CPU Type (\$08 = 6809, \$03 = 6309) \\
3-23 & Reserved for future definition \\
\hline
\end{tabular}
\end{center}
\end{table}

Once the CoCo has sent this packet, it shall go into WireBug Mode where it will wait for commands from the Server.

Along with notifying the Server that the CoCo is in WireBug mode, the packet also communicates the CoCo and processor type.	The Server should use this information as a cue to what type of processor is in the CoCo, and adjust accordingly.

\subsection{Read Registers Transaction (Server Initiated)}
Using this bi-directional transaction, the Server requests the contents of the registers from the CoCo.

\begin{table}[ht]
\caption{Read Registers Packet}
\begin{center}
\begin{tabular}{|ll|}
\hline
Byte & Value \\ \hline
0 & OP\_WIREBUG\_READREGS (\$52) \\
1-22 & 0 \\
23 & Checksum \\
\hline
\end{tabular}
\end{center}
\end{table}

Upon receipt of the packet, the CoCo shall compute the checksum of bytes 0-22 and compare its computed value against byte 23. If a checksum match does not occur, then the CoCo shall disregard the packet and send the following response packet:

\begin{table}[ht]
\caption{Checksum Mismatch Packet}
\begin{center}
\begin{tabular}{|ll|}
\hline
Byte & Value \\ \hline
0 & Checksum mismatch (243) \\
\hline
\end{tabular}
\end{center}
\end{table}

Upon receipt of this packet, the Server may elect to restart the transaction.

If a match occurs, the CoCo shall respond with the following packet:

\begin{table}[ht]
\caption{Read Registers Packet}
\begin{center}
\begin{tabular}{|ll|}
\hline
Byte & Value \\ \hline
0 & OP\_WIREBUG\_READREGS (\$52) \\
1 & Value of DP register \\
2 & Value of CC register \\
3 & Value of A register \\
4 & Value of B register \\
5 & Value of E register \\
6 & Value of F register \\
7 & Hi-byte of X register \\
8 & Lo-byte of X register \\
9 & Hi-byte of Y register \\
10 & Lo-byte of Y register \\
11 & Hi-byte of U register \\
12 & Lo-byte of U register \\
13 & Value of MD register \\
14 & Hi-byte of V register \\
15 & Lo-byte of V register \\
16 & Hi-byte of SP register \\
17 & Lo-byte of SP register \\
18 & Hi-byte of PC register \\
19 & Lo-byte of PC register \\
20-22 & Don't care \\
23 & Checksum \\
\hline
\end{tabular}
\end{center}
\end{table}

Upon receipt of the packet, the Server shall compute the checksum of bytes 0-22 and compare its computed value against byte 23. If a match occurs, the Server shall accept the packet�s payload.	If a match does not occur, then the Server may elect to restart the transaction.

\subsection{Write Registers Transaction (Server Initiated)}
Using this bi-directional transaction, the Server requests that the contents of the packet passed to the CoCo be written to its registers.

The following packet shall be sent to the CoCo:

\begin{table}[ht]
\caption{Write Registers Packet}
\begin{center}
\begin{tabular}{|ll|}
\hline
Byte & Value \\ \hline
0 & OP\_WIREBUG\_WRITEREGS (\$72) \\
1 & Value of DP register \\
2 & Value of CC register \\
3 & Value of A register \\
4 & Value of B register \\
5 & Value of E register \\
6 & Value of F register \\
7 & Hi-byte of X register \\
8 & Lo-byte of X register \\
9 & Hi-byte of Y register \\
10 & Lo-byte of Y register \\
11 & Hi-byte of U register \\
12 & Lo-byte of U register \\
13 & Value of MD register \\
14 & Hi-byte of V register \\
15 & Lo-byte of V register \\
16 & Hi-byte of SP register \\
17 & Lo-byte of SP register \\
18 & Hi-byte of PC register \\
19 & Lo-byte of PC register \\
20-22 & Don't care \\
23 & Checksum \\
\hline
\end{tabular}
\end{center}
\end{table}

Upon receipt of the packet, the CoCo shall compute the checksum of bytes 0-22 and compare its computed value against byte 23.	If a match occurs, the CoCo shall accept the packet�s payload, update its registers and send the following response packet:

\begin{table}[ht]
\caption{No Error Packet}
\begin{center}
\begin{tabular}{|ll|}
\hline
Byte & Value \\ \hline
0 & No error code (0) \\
\hline
\end{tabular}
\end{center}
\end{table}

If a checksum match does not occur, then the CoCo shall disregard the packet and send the following response packet:

\begin{table}[ht]
\caption{Checksum Mismatch Packet}
\begin{center}
\begin{tabular}{|ll|}
\hline
Byte & Value \\ \hline
0 & Checksum mismatch (243) \\
\hline
\end{tabular}
\end{center}
\end{table}

Upon receipt of this packet, the Server may elect to restart the transaction.

\clearpage
\subsection{Read Memory Transaction}
This bi-directional transaction is sent from the Server to the CoCo. uses this transaction to obtain memory values from the CoCo.

The following packet is sent from the Server:

\begin{table}[ht]
\caption{Read Memory Packet}
\begin{center}
\begin{tabular}{|ll|}
\hline
Byte & Value \\ \hline
0 & OP\_WIREBUG\_READMEM (\$4D) \\
1 & Hi-byte of starting memory location \\
2 & Lo-byte of starting memory location \\
3 & Count (0 < n < 23) \\
4-22 & Don't care \\
23 & Checksum \\
\hline
\end{tabular}
\end{center}
\end{table}

Upon receipt of the packet, the CoCo shall compute the checksum of bytes 0-22 and compare its computed value against byte 23. If a checksum match does not occur, then the CoCo shall disregard the packet and send the following response packet:

\begin{table}[ht]
\caption{Checksum Mismatch Packet}
\begin{center}
\begin{tabular}{|ll|}
\hline
Byte & Value \\ \hline
0 & Checksum mismatch (243) \\
\hline
\end{tabular}
\end{center}
\end{table}

Upon receipt of the Checksum Mismatch Packet, the Server may elect to restart the transaction.

If a match occurs, the CoCo shall validate the memory request. If byte 3 (count) is not between 1 and 22, then the CoCo shall send the following response packet:

\begin{table}[ht]
\caption{Illegal Response Packet}
\begin{center}
\begin{tabular}{|ll|}
\hline
Byte & Value \\ \hline
0 & Illegal number (16) \\
\hline
\end{tabular}
\end{center}
\end{table}

If the byte count is between 1 and 22, then the CoCo shall accept the packet�s payload and send the following response packet:

\begin{table}[ht]
\caption{Read Memory Response Packet}
\begin{center}
\begin{tabular}{|ll|}
\hline
Byte & Value \\ \hline
0 & OP\_WIREBUG\_READMEM (\$4D) \\
1..n & Memory from start to end \\
n+1..22 & Don't care \\
23 & Checksum \\
\hline
\end{tabular}
\end{center}
\end{table}

Upon receipt of the packet, the Server shall compute the checksum of bytes 0 to 22 and compare its computed value against byte 23.	If a match occurs, the Server shall accept the packet�s payload.	If a match does not occur, then the Server may elect to restart the transaction.

\subsection{Write Memory Transaction}
This bi-directional transaction is sent from the Server to the CoCo. uses this transaction to modify the contents of the CoCo�s memory.
The following packet is sent from the Server:

\begin{table}[ht]
\caption{Write Memory Packet}
\begin{center}
\begin{tabular}{|ll|}
\hline
Byte & Value \\ \hline
0 & OP\_WIREBUG\_WRITEMEM (\$6D) \\
1 & Hi-byte of starting memory location \\
2 & Lo-byte of starting memory location \\
3 & Count (n bytes where 0 < n < 20) \\
4-22 & Bytes to be modified \\
23 & Checksum \\
\hline
\end{tabular}
\end{center}
\end{table}

Upon receipt of the packet, the CoCo shall compute the checksum of bytes 0-22 and compare its computed value against byte 23. If a checksum match does not occur, then the CoCo shall disregard the packet and send the following response packet:

\begin{table}[ht]
\caption{Checksum Mismatch Packet}
\begin{center}
\begin{tabular}{|ll|}
\hline
Byte & Value \\ \hline
0 & Checksum mismatch (243) \\
\hline
\end{tabular}
\end{center}
\end{table}

If a match occurs, the CoCo shall validate the memory request. If byte 3 (count) is not between 1 and 19, then the CoCo shall send the following response packet:

\begin{table}[ht]
\caption{Illegal Response Packet}
\begin{center}
\begin{tabular}{|ll|}
\hline
Byte & Value \\ \hline
0 & Illegal number (16) \\
\hline
\end{tabular}
\end{center}
\end{table}

If the byte count is between 1 and 19, then the CoCo shall accept the packet�s payload, update its memory with the contents from the Server, and send the following response packet:

\begin{table}[ht]
\caption{No Error Packet}
\begin{center}
\begin{tabular}{|ll|}
\hline
Byte & Value \\ \hline
0 & No error code (0) \\
\hline
\end{tabular}
\end{center}
\end{table}


\subsection{Continue Execution Transaction}
This uni-directional transaction is sent from the Server to the CoCo. The Server uses this transaction to notify the CoCo that it should continue executing normally.

The following packet is sent from the Server:

\begin{table}[ht]
\caption{Continue Execution Packet}
\begin{center}
\begin{tabular}{|ll|}
\hline
Byte & Value \\ \hline
0 & OP\_WIREBUG\_GO (\$47) \\
\hline
\end{tabular}
\end{center}
\end{table}

Upon receiving this packet, the CoCo shall exit WireBug Mode and commence execution.	The Server shall commence responding to CoCo-initiated DriveWire transactions.

\section{Appendix A. Error Codes}
The error codes used by the Server to indicate an error condition mirror the same codes used in OS-9.	Here is a list of codes that should be used and the conditions that will trigger them:

\begin{enumerate}
\item \$F3 - CRC Error (if the Server�s computed checksum doesn�t match a write request from the CoCo)
\item \$F4 - Read Error (if the Server encounters an error when reading a sector from a virtual drive)
\item \$F5 - Write Error (if the Server encounters an error when writing a sector
\item \$F6	- Not Ready Error (if the a command requests accesses a non- existent virtual drive)
\end{enumerate}

\section{Appendix B. DriveWire Checksum Algorithm}
Even though DriveWire has been shown to provide very stable and reliable communications between CoCo and Server, a checksum algorithm has been employed to bring an added level of data integrity.

This new checksum algorithm is simpler and significantly better at tracking bit errors than the CRC algorithm employed in DriveWire Version 1, which used a fast but weak CRC method.
Designers of DriveWire Server software should implement the following checksum algorithm (presented in C source code) for computing a checksum for all sectors read and written to/from the CoCo.

\begin{lstlisting}
int calChecksum(unsigned char *ptr, int count)
{
	short Checksum = 0;
	while(--count)
	{
		Checksum += *(ptr++);
	}
	return (Checksum);
}
\end{lstlisting}

Note that WireBug uses an 8-bit checksum instead of a 16 bit one.
\end{document}
